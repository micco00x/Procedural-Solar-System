%%%%%%%%%%%%%%%%%%%%%%%%%%%%%%%%%%%%%%%%%
% Short Sectioned Assignment
% LaTeX Template
% Version 1.0 (5/5/12)
%
% This template has been downloaded from:
% http://www.LaTeXTemplates.com
%
% Original author:
% Frits Wenneker (http://www.howtotex.com)
%
% License:
% CC BY-NC-SA 3.0 (http://creativecommons.org/licenses/by-nc-sa/3.0/)
%
% Download template:
% Overleaf (https://www.overleaf.com/8746855dtrgkbkbjjhm)
%
%%%%%%%%%%%%%%%%%%%%%%%%%%%%%%%%%%%%%%%%%

%----------------------------------------------------------------------------------------
%	PACKAGES AND OTHER DOCUMENT CONFIGURATIONS
%----------------------------------------------------------------------------------------

\documentclass[paper=a4, fontsize=11pt]{scrartcl} % A4 paper and 11pt font size

%\usepackage[options]{nohyperref}  % This makes hyperref commands do nothing without errors
%\usepackage{url}  % This makes \url work
%\usepackage{hyperref}

\usepackage{graphicx}

\usepackage[T1]{fontenc} % Use 8-bit encoding that has 256 glyphs
\usepackage{fourier} % Use the Adobe Utopia font for the document - comment this line to return to the LaTeX default
\usepackage[english]{babel} % English language/hyphenation
\usepackage{amsmath,amsfonts,amsthm} % Math packages

\usepackage{lipsum} % Used for inserting dummy 'Lorem ipsum' text into the template

\usepackage{sectsty} % Allows customizing section commands
\allsectionsfont{\centering \normalfont\scshape} % Make all sections centered, the default font and small caps

\usepackage{fancyhdr} % Custom headers and footers
\pagestyle{fancyplain} % Makes all pages in the document conform to the custom headers and footers
\fancyhead{} % No page header - if you want one, create it in the same way as the footers below
\fancyfoot[L]{} % Empty left footer
\fancyfoot[C]{} % Empty center footer
\fancyfoot[R]{\thepage} % Page numbering for right footer
\renewcommand{\headrulewidth}{0pt} % Remove header underlines
\renewcommand{\footrulewidth}{0pt} % Remove footer underlines
\setlength{\headheight}{13.6pt} % Customize the height of the header

\numberwithin{equation}{section} % Number equations within sections (i.e. 1.1, 1.2, 2.1, 2.2 instead of 1, 2, 3, 4)
\numberwithin{figure}{section} % Number figures within sections (i.e. 1.1, 1.2, 2.1, 2.2 instead of 1, 2, 3, 4)
\numberwithin{table}{section} % Number tables within sections (i.e. 1.1, 1.2, 2.1, 2.2 instead of 1, 2, 3, 4)

%\setlength\parindent{0pt} % Removes all indentation from paragraphs - comment this line for an assignment with lots of text
\usepackage{indentfirst} % Indentation for all paragraphs

% Used for definitions:
\usepackage{amsthm}
\theoremstyle{definition}
\newtheorem{definition}{Definition}[section]

% To write algorithms in pseudocode:
\usepackage{algpseudocode}
\usepackage{algorithm}

% To put images side by side:
\usepackage{subcaption}

% URL available:
\usepackage{url}

% Don't use colon in algorithms lines:
\algrenewcommand\alglinenumber[1]{\footnotesize #1}

% Input/Output instead of Require/Ensure in algorithms pseudocode:
\renewcommand{\algorithmicrequire}{\textbf{Input:}}
\renewcommand{\algorithmicensure}{\textbf{Output:}}

%----------------------------------------------------------------------------------------
%	TITLE SECTION
%----------------------------------------------------------------------------------------

\newcommand{\horrule}[1]{\rule{\linewidth}{#1}} % Create horizontal rule command with 1 argument of height

\title{	
\normalfont \normalsize 
\textsc{Sapienza University of Rome} \\ [25pt] % Your university, school and/or department name(s)
\horrule{0.5pt} \\[0.4cm] % Thin top horizontal rule
\huge Interactive Graphics \\ % The assignment title
\large Final Project: Procedural Solar System \\
\horrule{2pt} \\[0.5cm] % Thick bottom horizontal rule
}

\author{Michele Cipriano, Ivan Bergonzani} % Your name

\date{\normalsize\today} % Today's date or a custom date

\begin{document}

\maketitle % Print the title

%----------------------------------------------------------------------------------------

\section{Introduction}

The theme of the project is about the generation of a solar system using
procedural graphics. The aim is to study in depth procedural meshes developing
a system that ease the creation of pseudorandomly generated planets.

The program is characterized by having a solar system composed by four planets,
a star and a satellite (the Sun, the first four planets of our solar system and the
Moon, orbiting the Earth). The meshes are generated using Perlin noise, each
planet is composed by multiple chunks which change definition depending on
the distance, allowing high performances even with a high number of details.
The system supports multiple lights and textures, managed in shaders developed
with GLSL. Objects are animated with rotation around themselves and
revolution around the parents' object (the Sun for the planets, the Earth for
the Moon). A simple flying system has been implemented with the help of
THREE.js \cite{threejs}, which simplifies the development of the whole program as well.

The program obtains optimal performances running smoothly both when the
camera is far away, where all the objects can be seen altogether, and when the camera
is near a planet, where seas, plains and mountains can be seen in detail.

%----------------------------------------------------------------------------------------

\section{Perlin Noise}

The main character of the project is the Perlin noise, resposible for the
generation of the pseudorandom meshes of the planets.

Against regular noise, where each pixel has assigned a random value between 0
and 1, changes occur gradually in Perlin noise, as in natural terrains.

To generate realistic terrains it's necessary to add multiple noise maps
together (called octaves), each of which has different amplitude and frequency,
determined by the parameters lacunarity and persistance. These two are
responsible for the number of small features of the terrain and how much
these features change the shape of it.

Noise values are used to determine the height of the terrain for each vertex of
the planet. To do this without loosing continuity between adjacent vertices, a 3D
Perlin noise function has been implemented passing as parameters the points of
a hypothetical sphere. Hence, for a point $(x, y, z)$ on the sphere there is a
corresponding height $h$.
The algorithm for the generation of the heights is
Algorithm \ref{algorithm:noise3D}. Note that the number of octaves, the persistance,
the lacunarity and the scale are constant values (choosen before the execution
of the function). The scale value directly affects the frequency of the function.
An external \textsc{Perlin} function is used to compute the gradient noise.

\begin{algorithm}
	\caption{Noise height generator.}
	\label{algorithm:noise3D}
	\begin{algorithmic}[1]
		\Function{noiseHeight}{$x$, $y$, $z$}
			\State $amplitude \gets 1$
			\State $frequency \gets 1$
			\State $h \gets 0$
			\For {$oct \gets 1$ \textbf{to} $octaves$}
				\State $sampleX \gets x$ / $scale * frequency$
				\State $sampleY \gets y$ / $scale * frequency$
				\State $sampleZ \gets z$ / $scale * frequency$
				
				\State $h \gets h$ + \textsc{Perlin}$(sampleX, sampleY, sampleZ) * amplitude$
				
				\State $amplitude \gets amplitude * persistance$
				\State $frequency \gets frequency * lacunarity$
			\EndFor
			\State \Return $h$
		\EndFunction
	\end{algorithmic}
\end{algorithm}

%----------------------------------------------------------------------------------------

\section{Planet Generation}

The first thing that comes to mind when creating a planet is to use a spherical
geometry. Nevertheless the sphere has several drawbacks when dealing with
chunks, when changing the level of detail (to preserve performances) and, in
particular, when textures are handled. These are, in fact, stretched at the poles,
creating a bad graphical effect that ruins the realism of the whole planet.

A solution is to use a cube and ``spherify'' it by modifying its vertices.
In this way the triangles of the geometry will have a similar size and the
stretching of the texture will be solved.

While this method solves the problem, a cube geometry is not suitable for chunks
and level of details (LOD). In fact, it's
necessary to divide the faces of the cube in multiple parts (chunks),
and change the resolution of these depending on the distance from the camera.
This is exactly what happens in graphic engines when huge terrains are
rendered.

A solution is to compose the cube using multiple planes, each plane will
behave like a chunk and will change its resolution dinamically using LODs.
In particular, each face of the cube is divided in $N*N$ chunks, hence, a
planet is composed by $6*N*N$ chunks.

Vertices of the chunks are initialized using Perlin noise. The height of each
vertex w.r.t. the center of the earth is determined by first spherifying the
vertex itself and then computing the noise height in that point.

%----------------------------------------------------------------------------------------

\section{Chunks}

As said before, planets are made of parts called chunks. Each chunk, in the
implementation, is actually an element of a LOD object, multiple chunks of a
LOD determine how the resolution of the terrain changes with distance.
In this case, even if it's defined with a plane geometry, the chunk it's
a custom geometry which follows the curvature of the planet.

Choosing the number of chunks which compose the planet is important both to
have high details and to have good performances. In particular, a low number
of chunks improves the performances when the camera is far away; this is due
to the fact that all the meshes of all the planets in the viewport must be
rendered. A high number of chunks increases the number of details, but
must not be too high, otherwise the performances will be bad when moving away
from the planet. The number of chunks can be decreased by increasing the
number of vertices of each chunk obtaining the same level of details.
Nevertheless, this number shouldn't be
too high otherwise the rendering will be slow when approaching the planet. 

One of the problem of creating a terrain with multiple meshes instead of just
one is the wrong value assigned to the normals at the border. This is due to
the fact that normals are computed considering the adjacent vertices, which
are not available for the vertices at the borders of the mesh.

A simple solution is to define an extended chunk which overlaps the
chunk that will be used for the planet, but contains a new, bigger border.
In this way the value of the normals of the extended chunk, with the exception
of its border, can be copied back to the original chunk, which will then have
correct normals also at the border. This eliminates the difference of colors
at borders of the chunks, making the planet look like a single piece.

%----------------------------------------------------------------------------------------

\section{LOD}

Dividing the planet in multiple pieces is not enough to obtain good
performances, this is why level of details (LOD) are used when building
terrains.

The idea is simple, instead of composing the planet with multiple chunks,
the planet is composed by multiple LOD objects that contain the same
chunk with different amounts of vertices. Thus, the same chunk is ``computed
more times'' with different details. Then, at rendering time, each LOD
object is updated depending on the position of the camera, in this way it's
possible to gradually change the amount of vertices of the mesh while
approaching the planet (Figure \ref{fig:LOD-venus}).

Choosing the right amount of LODs' chunks its important to have good
performances both while being far away from a planet and while approaching it.
In particular, this allows the GPU to handle a lower number of vertices
for the planets distant from the camera, making it possible to have more
resources for the closer meshes, that needs to be more detailed.

\begin{figure}
	\centering
	\begin{subfigure}{.3\textwidth}
		\centering
		\includegraphics[width=1.0\linewidth]{images/venus_d0.png}
	\end{subfigure}
	\begin{subfigure}{.3\textwidth}
		\centering
		\includegraphics[width=1.0\linewidth]{images/venus_d1.png}
	\end{subfigure}
	\begin{subfigure}{.3\textwidth}
		\centering
		\includegraphics[width=1.0\linewidth]{images/venus_d2.png}
	\end{subfigure}
	\caption{Level of detail increases while approaching Venus.}
	\label{fig:LOD-venus}
\end{figure}

%----------------------------------------------------------------------------------------

\section{Lights}

THREE.js simplifies a lot the management of the lights even when a custom
shader is used, like in the case of this program. Here two point lights are used
in the scene, one for the Sun and one for the Moon (Figure \ref{fig:earth-moon-light}). Lights are directly added
as a child of the Planet in the hierarchical structure, making it easy to
manage their movement. Since the custom shader increases the flexibility in
the development,
the parameter ``distance'' of each light has been used to determine for how much
distance the lights affects the objects, instead of being used to compute the
attenuation value.

The shader supports also the light emitted by the planet, used to make the Sun,
the Moon and Venus brighter.

\begin{figure}
	\centering
	\includegraphics[scale=0.25]{images/earth_moon.png}
	\caption{The light of the Moon affects the color of the Earth.}
	\label{fig:earth-moon-light}
\end{figure}

%----------------------------------------------------------------------------------------

\section{Fly Control}

User interaction has been developed with a tool of THREE.js called FlyControl,
which has been used for the creation of a simple flying system that allows to
move around the solar system.

FlyControl manages the position and the orientation of the camera taking
movements from the keyboard and the mouse.

%----------------------------------------------------------------------------------------

\section{Animation}

The solar system is composed by the Sun, positioned at the center of the scene,
four planets orbiting around the Sun which have similar characteristics to
Mercury, Venus, the Earth and Mars, and the Moon, which is orbiting around the
Earth. Moreover all these objects are rotating around themselves (Figure
\ref{fig:solar-system}).

The implementation of the orbital revolution is quite straightforward, it's necessary
in fact, to slightly change the position of the planets following a
circular path. Since this movement must affects also the children in the scene,
nothing else has to be done.

On the other hand, the rotation has to be
managed in a different way, otherwise the children will rotate with the
rotation of the parent. To solve this problem, instead of adding the LOD objects
directly to the planet, a pivot object is used. This object is added to the
planet and it's, in turn, parent of all the LOD objects of the planet itself.
Thus, instead of
rotating the planet, only the pivot is rotated, preserving the position of
the children.

\begin{figure}
	\centering
	\includegraphics[scale=0.22]{images/solar_system.png}
	\caption{The planets are rotating around the Sun while the Moon is
		rotating around the Earth.}
	\label{fig:solar-system}
\end{figure}

%----------------------------------------------------------------------------------------

\section{Shaders and Textures}

As said before, the shaders are made by hand. This design choise is due to
the fact that it becomes easier to manage the textures of the terrain.

In fact, the texture changes on the height of the fragment, which was previously
determined using Perlin noise. This is necessary to assign, for the example
of the Earth, sea textures to low
heights, grass and mountain textures, or snow, to higher heights, creating
more realistic planets and differentiating seas by plains and mountains.

Moreover, a texture for the ``background'' is used, mapping a starfield on a
spherified cube, instead of a sphere, to solve, again, stratching of the image at the poles.

%----------------------------------------------------------------------------------------

\section{How to Use}

Before launching the HTML file on the browser, it's necessary to execute a
server to load external files directly from the Javascript files. This can be
easily done with Python: \texttt{python3 -m http.server}.

The server will be created on port 8000, if available. It's then possible
to load the HTML file from \texttt{http://localhost:8000/main.html}.

The program has as inputs the inputs managed by the fly controller, which are:

\begin{itemize}
	\item \texttt{W}, \texttt{A}, \texttt{S}, \texttt{D}, mouse: move
	\item \texttt{R}, \texttt{F}: up/down
	\item \texttt{Q}, \texttt{E}: roll
	\item up, down: pitch
	\item left, right: yaw
\end{itemize}

%----------------------------------------------------------------------------------------

\section{Conclusion}

Procedural graphics in the last years have seen a big adoption in both videogames
and movie industry, this project is just an example of what can be done using
a simple tool like Perlin noise.

The creation of a complex planet full of details is not straightforward, even
by using tools made available by a library like THREE.js. Optimizing the structure of the program, in particular using chunks and
level of details is a must to deal with large terrains and with a high number
of meshes.

THREE.js simplifies the management of objects like planets and lights, making
it easy to manage the scene; moreover, having a large number of tools, like
different geometries and controls, speeds up the development of the project,
allowing the developers to concentrate more on the details of the program
than the structure of it.

%----------------------------------------------------------------------------------------

\newpage
\bibliography{bibliography}
\bibliographystyle{ieeetr}

%----------------------------------------------------------------------------------------

\end{document}
\grid
\grid
